\subsection{Approximation by Mateusz Bysiek}

\subsubsection{Class variables}
The same as in case of accurate algorithm.

\subsubsection{Algorithm}
Development of approximations of NP problems is usually a consequence 
of deep understanding of the problem, and extensive experience in solving it using
some kind of accurate algorithm.

During my experiments, I have discovered that depth-first construction/search of DAG is far more
efficient than breadth-first, when it comes to actual time needed to find the best result. It is so
because we need to find any best result, not all of them. In case of breadth-first, 

I have put all above optimizations into the accurate algorithm, and I have created approximate one
by simply limiting number of possible options to be explored. I have limited number of explored
options to $n^2$, where $n$ is number of elements initially on board. In short, after line 60, I
added this:

\begin{minted}[linenos,numbersep=5pt]{ruby}
   if hashed.element_count > square(elements.length)
      break
\end{minted}

\subsubsection{Optimizations}
I have also noticed that when I use depth-first search in such way as described in the accurate
algorithm's pseudo-code, the tree is searched from 'right' to 'left', and when algorithm is
analyzing any state x, that means that all states to the right of it are already explored.

I have used this to develop optimization which removes obsolete explored branches, i.e. those
branches that do not contain currently known best state. This greatly reduced memory usage.

I have also noticed that list of pieces currently on board can be easily converted to integral type
(if number of pieces initially on board is less than maximum number of bits stored in that type).
Keeping all lists of pieces (with exception of the 'reference' list used to convert them back) as
integers also greatly decreased memory usage, and allowed creation of a very cheap (in terms of time
and memory) hash table (used as the field 'hashed'), which has constant access/storage/search time
equal to $\log (k)$ where $k$ is number of pieces initially on board.

The above (and some other, minor) optimizations caused the algorithm to be able to solve problems
with at most 64 not-invalid pieces. This means that if length of 'checked' exceeds 64, the program
will crash, when either accurate algorithm, or approximate algorithm created by me (M. Bysiek) is used.

\subsubsection{Complexity}
Due to hard limit of checking at most $n^2$ states, and assuming that checking each state takes
polynomial time, such approximate algorithm is also polynomial.
