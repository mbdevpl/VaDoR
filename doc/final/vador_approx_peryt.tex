\subsection{Approximation by Stanisław Peryt}

\subsubsection{Input Data}
collection of domino pieces from input file
\subsubsection{Description and Class Variables}
This algorithm is based on the idea of Greedy algorithms. I assume some additional classification
which is done only once for each element. For classification purposes each dominoPiece should
contain some additional fields:

\begin{minted}[linenos,numbersep=5pt]{c++}
boolean isIndependent; //can be removed even from empty board, so it does not have to 
   // be removed immediately
boolean canBeRemoved; //initially false. Changes to true, when fulfills removal conditions
list<waitingFor>; // list of pieces for which it is waiting;
list<waitForMe>; // list of pieces waiting for it;
\end{minted}

I decided to add this classification, because in greedy strategy, we could remove pieces immediately
when they fulfill removal requirements. But it may happen, that there are some pieces, for which
pieces we already removed are necessary to fulfill removal requirements. When we have this
classification based on scopes of domino pieces, we can wait with removing, and eliminate at least
some of those situations described above, providing better local optimal choices.

For this algorithm we also need:
board – contains all domino pieces that have remained on the board
roots – contains possible starting pieces, i.e. containing zero
neighbors – contains domino pieces, which are neighbors of deleted piece, i.e. pieces alongside empty fields
\subsubsection{Algorithm}
\begin{enumerate}[noitemsep,nolistsep]
   \item Create board from input
   \item Find possible roots and assign them to neighbors
   \item for each dominoPiece in board assign scopes and waiting property
   \begin{enumerate}[noitemsep,nolistsep]
      \item determine scopes based on numbers given on dominoPiece
      \item determine dominoPieces on scopes and assign them to the waitForMe list
      \item tell those pieces to assign this dominoPiece on their waitingFor list
   \end{enumerate}
   \item stop=false;
   \item while(!stop)\{
   \begin{itemize}[noitemsep,nolistsep]
     \item for each dominoPiece in neighbors\{
         \begin{enumerate}[noitemsep,nolistsep]
            \item if dominoPiece can be removed \{
            \begin{enumerate}[noitemsep,nolistsep]
               \item if (domino Piece isIndependent) and (list of pieces it is waiting for is not empty) \{ \\
                  wait with removal; \\
               \}
               \item else
                  remove it(from neighbors and board) and tell waiting pieces, if there were any waiting
                  for this piece, that it does not need waiting any more. (i.e. delete piece from waiting
                  list)
            \end{enumerate}
            \}
         \end{enumerate}
         \} \\
         determine new neighbors into temp (each removed piece can produce at most 6 new neighbors)
     \item if temp equals neighbors\{
         \begin{itemize}[noitemsep,nolistsep]
            \item check if (something can be removed): remove first;
            \item else stop=true;
         \end{itemize}
         \}
     \item else neighbors=temp;
   \end{itemize}
   \}
   \item calculate score
\end{enumerate}
\subsubsection{Estimation of complexity}
Partial Complexities:

$n$-complexity of classification

$n^2$-complexity of part at “while(!stop) loop”

In order to associate attributes for classification we need to iterate once through whole collection
which is of size n. Then in the while loop we are trying to remove pieces. In worst case we delete
only one piece in one iteration which gives us at most n runs of while and inside while loop, we
iterate through neighbors collection which is always smaller than n, but it is n dependent.
Therefore for each n, algorithm executes number of operations proportional to n, which gives us
quadratic time complexity.

Final worst case complexity : $n^2 + n$

\subsubsection{Changes after tests}

Classification feature was completely removed from the algorithm due to the fact that its complexity
exceeded estimations.
