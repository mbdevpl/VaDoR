\subsection{Class variables}
Main class of the program is called domino\_problem. Rough list of fields of the class:

\begin{minted}[linenos,numbersep=5pt]{ruby}
# size_t is an integral type
# elements_t is a list of domino pieces
# board_t is a two-dimensional array of references to the halves of domino pieces
   size_t width
   size_t height
   # collection of pieces
   elements_t elements
   # collection of halves of pieces that come from 'elements' field
   board_t board
   # pieces currently on the board
   elements_t on_board
   # possible to remove in the next turn
   elements_t possible
   # not longer on board, removed in the previous turns
   elements_t removed
   # algorithm does not know anything about these pieces
   elements_t unresolved
   # possible to remove if other pieces are placed right
   elements_t checked
   # impossible to remove due to size of the board and/or dependencies on other invalid pieces
   elements_t invalid
\end{minted}

There is also a derived class, called domino\_problem\_solver, which contains several other
variables, which store data about every state analyzed by the algorithm.

\begin{minted}[mathescape,linenos,numbersep=5pt,gobble=0,framesep=2mm]{ruby}
# state_list_t is a list of states (domino_problem-s)
# dag_t is a directed acyclic graph
# hash_t is a radix-based hashing table for integral types
   # contains unexplored states
   state_list_t unexplored
   # contains current tree of states (some explored, some not) connected
   #  according to the rules given below
   dag_t tree
   # contains explored states
   hash_t hashed
   # reference to currently best known state (state with max. number of removed pieces)
   domino_problem best_state
\end{minted}

%\pagebreak[4]

\subsection{Algorithm}
%Because of large number of loops involved in the algorithm, 
%I will provide pseudo-code, and will write down 
%a description of instructions.
% \begin{enumerate}[noitemsep,nolistsep]
%    \item Input data is read: width and height.
%    \item All pieces present on board are stored into 'elements' field
%    this list does not change with time
%    \item 'board' is generated for convienience.
%    \item 'elements' are copied into other lists: 'on\_board' and 'unresolved'
%    \item 'unresolved' are partitioned into 'checked' and 'invalid'
%    \item add current problem to the graph
%    \item CHECKING: each piece from 'checked' is examined, and if it can be removed 
%    from the board it is copied to 'possible'
%    \item for each 'possible' piece, new object of class domino\_problem is created
%    via copy constructor
%    \begin{enumerate}[noitemsep,nolistsep]
%       \item from each copy of current domino\_problem, the corresponding piece of 'possible'
%       is moved to 'removed', and the same piece is deleted from 'checked', 'on\_board' 
%       and 'board'.
%       \item all lists of 'possible' from current domino\_problem copies are cleared
%       \item add all copies to the graph with exception of those that already exist
%       in the graph, because removing first piece A and then B gives the same resulting state 
%       as removing first piece B and then A. Calculating all possibilities with duplicates would
%       result in factorial complexity of the algorithm ($n!$).
%       \item connect current problem to all of the copies
%       \item for each copy, perform all steps starting from CHECKING
%    \end{enumerate}
%    \item after the graph is constructed, find the longest path, starting from 
%    initial domino\_problem.
% \end{enumerate}

Below is an outline of the algorithm for deterministic RAM:

\begin{minted}[linenos,numbersep=5pt]{ruby}
   # 1. read input data
   width = from input
   height = from input
   elements = all pieces present on board # this list does not change with time
   # board is generated using 3 above variables
   board = generate_board(elements, width, height) 
   # 'elements' are copied into 'on_board' and 'unresolved'
   on_board = unresolved = elements
   # 'unresolved' are partitioned into 'checked' and 'invalid'
   resolve(unresolved, checked, invalid)
   # current state is copied from initial data of the problem, and it is put into the tree,
   #  becoming its root
   domino_problem current_state = self
   tree += current_state
   unexplored += current_state
   best_state = current_state
   # all pieces that can be removed in current state are found,
   #  by analyzing the 'checked' list
   find_possible(current_state)
   
   # 2. build the tree of states
   while unexplored is not empty do
      domino_problem s = unexplored.last_element
      unexplored -= s
      
      state_list_t new_states
      for element e in s.possible
         # make a copy of 's', with the exception of list of possible pieces
         domino_problem pr = s 
         # from copy of current domino\_problem, the corresponding piece is added to 'removed',
         # and the same piece is deleted from 'checked' and 'on\_board'
         pr.on_board -= e
         pr.checked -= e
         pr.removed += e
         find_possible(pr)
         new_states += pr
      end
       
      # sort new_states ascending, using length of list of 'possible' as criteria
      sort(new_states)
      
      # add all new states to 'tree', with exception of those that already exist
      # in 'tree', because removing first piece A and then B gives the same resulting state 
      # as removing first piece B and then A. Calculating all possibilities with duplicates would
      # result in factorial complexity of the algorithm (n!).
      for domino_problem st in new_states
         if hashed does not contain st
            hashed += st
            unexplored += st
            # add the new state to 'tree', as a child of 's'
            add_child(tree, s, st)
         end
      end
      
      if best_state.on_board.length > s.on_board.length then
         best_state = s
         if best_state.on_board.length == invalid.length then
            break
      end
   
      # this optimization causes complexity to go beyond 2^n, but it is used only if
      # algorithm would have ended computation otherwise due to no memory
      if running out of memory then
         hashed.remove_all
      end
   end
   
   # 3. return final result
   return best_state
\end{minted}

\subsection{Estimation of complexity}

\subsubsection{Number of states}
Let us assume that we have a set of $n$ domino pieces that are arranged on a board. Then, 
let us assume that we are taking away pieces from the board, one at a time. If we consider each 
state after removing a piece, how many different states do we have? For a given state, for each piece, 
we have a choice: the piece either is on the board, or it is not. In total we have $2$ possible scenarios
for one piece, and $n$ pieces, which gives us exactly $2^n$ different states.

\subsubsection{Relations between the states}
We can construct a following directed acyclic graph (DAG) $G$:
\begin{enumerate}[noitemsep,nolistsep]
  \item each vertex $V$ represents a state (configuration of pieces on the board)
  \item each directed edge $E$ represents the relation between states. Edges 
  are positioned in such way, that start point is at state $V_1$, and such that 
  the end point of the edge is connected to the state $V_2$ that can be obtained 
  directly from state $V_1$ by removing one domino piece from the board.
\end{enumerate}

\subsubsection{Proof that VaDoR is NP}

Below, outline of a polynomial-time algorithm for a non-deterministic RAM:

\begin{minted}[linenos,numbersep=5pt]{ruby}
input: b # list of elements on board

var r # list of removed elements, initially empty

for each element e in b
   # magical 'if' that knows when putting current element to the list 
   #  is the best possible option - such 'if-better' is allowed for non. det. RAMs
   if-better 
      add e to r
   else
      next

return r

\end{minted}

\subsubsection{Final estimation of complexity}
Having a DAG $G$, the domino problem is equivalent to finding the longest path of DAG. 
Unfortunately, graph has to be constructed at runtime, and that takes time. 

Since, for each state, the calculation of related states will be performed (and there are $2^n$
states in total) I estimate the complexity of accurate algorithm as exponential. Due to this maximum
number of states, I also informally classify this algorithm as NP-complete, however, it is only my
intuition.

After the DAG is constructed, by finding its longest path, an optimal solution is obtained. But,
because the DAG is not given as input, but rather constructed during algorithm's operation, finding
longest path can be done while constructing the DAG, therefore no extra time is needed for it.

My estimations indicate that calculation of possible sub-states for a given single state takes
polynomial time. Therefore, the total complexity of the accurate algorithm should be roughly $ 2^n $
In practice, however, I have noticed that it is not such, because problems encountered in practice
tend to have far less possible choices, due to strong inter-piece dependencies.
 