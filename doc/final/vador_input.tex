\subsubsection{Text file}
Described in the documentation provided by the laboratories supervisor.

\subsubsection{Our own format}
XML file.
\begin{minted}[linenos,numbersep=5pt]{xml}
<domino_board width="5" height="4">
    <piece x="0" y="0" orientation="vertical" value1="2" value2="1" />
    <piece x="0" y="2" orientation="vertical" value1="3" value2="2" />
    <piece x="1" y="0" orientation="horizontal" value1="1" value2="0" />
    <piece x="1" y="1" orientation="vertical" value1="3" value2="0" />
    <piece x="1" y="3" orientation="horizontal" value1="0" value2="0" />
    <piece x="2" y="1" orientation="vertical" value1="1" value2="1" />
    <piece x="3" y="0" orientation="horizontal" value1="3" value2="3" />
    <piece x="3" y="1" orientation="vertical" value1="0" value2="2" />
    <piece x="3" y="3" orientation="horizontal" value1="2" value2="2" />
    <piece x="4" y="1" orientation="vertical" value1="4" value2="4" />
</domino_board>
\end{minted}

\noindent Description of the \verb|piece| tag:
\begin{itemize}[noitemsep,nolistsep]
  \item x - coordinate, from zero, increasing from the right to the left side of the board
  \item y - coordinate, from zero, increasing from the top to the bottom side of the board
  \item orientation:
   \begin{itemize}[noitemsep,nolistsep]
   \item \emph{horizontal} - the piece starts at $(x,y)$ and ends at $(x+1,y)$
   \item \emph{vertical} - the piece starts at $(x,y)$ and ends at $(x,y+1)$
   \end{itemize}
  \item value1 - value of the beginning of the piece i.e. value at $(x,y)$
  \item value2 - value of the end of the piece i.e. location depends on the orientation
\end{itemize}

\vspace{10pt}

If a tag \verb|removed_pieces| is present inside the \verb|domino_board| tag, it is ignored. If an
attribute \verb|order| is attached to the \verb|piece| tag, it is also ignored. That way the output file 
can be used as an input, and pieces can be rearanged and put back on the board using a text editor, without
worrying about problems with parsing.
